\documentclass[12pt,a4paper]{article}
\usepackage[utf8]{inputenc}
\usepackage[brazil]{babel}
\usepackage{graphicx}
\usepackage{amssymb, amsfonts, amsmath}
\usepackage{float}
\usepackage{enumerate}
\usepackage[top=2.5cm, bottom=2.5cm, left=1.25cm, right=1.25cm]{geometry}

\begin{document}
\pagestyle{empty}

\begin{center}
  \begin{tabular}{ccc}
    \begin{tabular}{c}
      \includegraphics[scale=0.25]{../../biblioteca/imagem/brasao-de-armas-brasil} \\
    \end{tabular} & 
    \begin{tabular}{c}
      Ministério da Educação \\
      Universidade Federal dos Vales do Jequitinhonha e Mucuri \\
      Faculdade de Ciências Sociais, Aplicadas e Exatas - FACSAE \\
      Departamento de Ciências Exatas - DCEX \\
      Disciplina: Introdução à Ciência da Computação \quad Semestre: 2024/2\\
      Prof. Dr. Luiz C. M. de Aquino\\
    \end{tabular} &
    \begin{tabular}{c}
      \includegraphics[scale=0.25]{../../biblioteca/imagem/logo-ufvjm} \\
    \end{tabular}
  \end{tabular}
\end{center}

\begin{center}
  \textbf{Lista I}
\end{center}

\begin{enumerate}
  \item Faça um programa que leia dois números e informe qual é o menor deles.
  \item Faça um programa que leia três números e mostre-os em ordem decrescente.
  \item Faça um programa para a leitura de duas notas parciais de um aluno. O programa deve 
  calcular a média alcançada pelo aluno e apresentar:
    \begin{itemize}
      \item a mensagem ``Aprovado com Distinção'', se a média for igual a dez;
      \item a mensagem ``Aprovado'', se a média alcançada for maior ou igual a sete e menor do que dez;
      \item a mensagem ``Reprovado'', se a média for menor do que sete;
    \end{itemize}
  \item Faça um programa que leia seis números racionais dados por $a$, $b$, $c$, $d$, $e$ e $f$. 
  Em seguida, o programa deve analisar o sistema formado por:
  $$
  \begin{cases}
    ax + by = e \\
    cx + dy = f
  \end{cases}
  $$
  
  Se o sistema tiver solução única, o programa deve informá-la. Caso contrário, o programa deve apresentar
  a mensagem: ``não foi possível achar uma solução única para o sistema''.

\end{enumerate}

\end{document}