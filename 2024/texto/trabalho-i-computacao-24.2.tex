\documentclass[12pt,a4paper]{article}
\usepackage[utf8]{inputenc}
\usepackage[brazil]{babel}
\usepackage{graphicx}
\usepackage{amssymb, amsfonts, amsmath}
\usepackage{float}
\usepackage{enumerate}
\usepackage[top=2.5cm, bottom=2.5cm, left=1.25cm, right=1.25cm]{geometry}

\DeclareMathOperator{\sen}{sen}
\DeclareMathOperator{\tg}{tg}

\begin{document}
\pagestyle{empty}

\begin{center}
  \begin{tabular}{ccc}
    \begin{tabular}{c}
      \includegraphics[scale=0.25]{../../biblioteca/imagem/brasao-de-armas-brasil} \\
    \end{tabular} & 
    \begin{tabular}{c}
      Ministério da Educação \\
      Universidade Federal dos Vales do Jequitinhonha e Mucuri \\
      Faculdade de Ciências Sociais, Aplicadas e Exatas - FACSAE \\
      Departamento de Ciências Exatas - DCEX \\
      Disciplina: Introdução à Ciência da Computação \quad Semestre: 2024/2\\
      Prof. Dr. Luiz C. M. de Aquino\\
    \end{tabular} &
    \begin{tabular}{c}
      \includegraphics[scale=0.25]{../../biblioteca/imagem/logo-ufvjm} \\
    \end{tabular}
  \end{tabular}
\end{center}

\begin{center}
  \textbf{Trabalho I}
\end{center}

  Suponha que você vai usar uma matriz $10\times 10$ para representar o tabuleiro 
  do jogo Batalha Naval. Nessa matriz, os elementos iguais a $0$ vão representar 
  posições desocupadas, enquanto que elementos iguais a $1$, $2$ ou $3$ vão representar
  posições ocupadas por embarcações.
  
  Considere que existem três tipos de embarcação nesse
  jogo. O Tipo I é composto pela matriz $1\times 1$ dada por $\begin{bmatrix} 1 \end{bmatrix}$.
  O Tipo II é composto pela matriz $1\times 2$ (ou $2\times 1$) dada por 
  $\begin{bmatrix} 2 \\ 2 \end{bmatrix}$ (ou $\begin{bmatrix} 2 & 2 \end{bmatrix}$).
  Por fim, o Tipo III é composto pela matriz $2\times 2$ dada por 
  $\begin{bmatrix} 3 & 3\\ 3 & 3 \end{bmatrix}$.
  
  Faça um algoritmo em Python que recebe uma matriz $10\times 10$ representando 
  o tabuleiro de um jogador. Considere que o jogador tem 3 embarcações do Tipo I,
  2 do Tipo II e 1 do Tipo III. O seu algoritmo deve procurar nessa matriz as posições
  ocupadas pelas embarcações e informá-las. Por exemplo, suponha que seja dada a seguinte
  matriz:
  $$
  \begin{bmatrix}
    0 & 0 & 0 & 0 & 0 & 0 & 0 & 0 & 0 & 0 \\
    0 & 0 & 0 & 0 & 0 & 0 & 2 & 2 & 0 & 0 \\
    0 & 0 & 0 & 0 & 0 & 0 & 0 & 1 & 0 & 0 \\
    0 & 0 & 1 & 0 & 0 & 0 & 0 & 0 & 0 & 0 \\
    0 & 0 & 0 & 0 & 0 & 0 & 0 & 0 & 0 & 0 \\
    0 & 0 & 0 & 0 & 0 & 0 & 0 & 0 & 0 & 0 \\
    0 & 0 & 0 & 0 & 3 & 3 & 0 & 0 & 0 & 0 \\
    0 & 0 & 0 & 0 & 3 & 3 & 0 & 0 & 1 & 0 \\
    0 & 0 & 2 & 0 & 0 & 0 & 0 & 0 & 0 & 0 \\
    0 & 0 & 2 & 0 & 0 & 0 & 0 & 0 & 0 & 0
  \end{bmatrix}
  $$  
  
  O seu algoritmo deve informar que:
  
  \begin{itemize}
    \item as embarcações do Tipo I estão nas posições $(4,\, 3)$, $(3,\, 8)$ e
    $(8,\, 9)$;
    \item as embarcações do Tipo II estão nas posições $(2,\, 7)$ e $(9,\, 3)$;
    \item a embarcação do Tipo III está na posição $(7,\, 5)$.
  \end{itemize}

\end{document}