\documentclass[12pt,a4paper]{article}
\usepackage[utf8]{inputenc}
\usepackage[brazil]{babel}
\usepackage{graphicx}
\usepackage{amssymb, amsfonts, amsmath}
\usepackage{float}
\usepackage{enumerate}
\usepackage[top=2.5cm, bottom=2.5cm, left=1.25cm, right=1.25cm]{geometry}

\begin{document}
\pagestyle{empty}

\begin{center}
  \begin{tabular}{ccc}
    \begin{tabular}{c}
      \includegraphics[scale=0.25]{../../biblioteca/imagem/brasao-de-armas-brasil} \\
    \end{tabular} & 
    \begin{tabular}{c}
      Ministério da Educação \\
      Universidade Federal dos Vales do Jequitinhonha e Mucuri \\
      Faculdade de Ciências Sociais, Aplicadas e Exatas - FACSAE \\
      Departamento de Ciências Exatas - DCEX \\
      Disciplina: Introdução à Ciência da Computação \quad Semestre: 2024/2\\
      Prof. Dr. Luiz C. M. de Aquino\\
    \end{tabular} &
    \begin{tabular}{c}
      \includegraphics[scale=0.25]{../../biblioteca/imagem/logo-ufvjm} \\
    \end{tabular}
  \end{tabular}
\end{center}

\begin{center}
  \textbf{Lista III}
\end{center}

\begin{enumerate}
  \item Faça um programa que peça um número natural $n$ e imprima
  na tela todos os múltiplos dele que são menores do que $n^2$.
  \item Faça um programa que peça um número natural $n$ e imprima na tela os números 
  quadrados perfeitos que são menores do que $n$.
  \item Faça um programa que leia dois números naturais $n$ e $m$ (com $n < m$). Em seguida,
  o programa deve imprimir o produto entre todos os números ímpares no intervalo $[n,\, m]$.
  \item Faça um programa que peça a quantidade de alunos de uma turma e em seguida
  as suas alturas. O programa deve calcular a média de altura dessa turma.
  \item Faça um programa que determine todas as soluções naturais da equação:
  $$7x + 4y = 142$$ 

\end{enumerate}

\end{document}