\documentclass[12pt,a4paper]{article}
\usepackage[utf8]{inputenc}
\usepackage[brazil]{babel}
\usepackage{graphicx}
\usepackage{amssymb, amsfonts, amsmath}
\usepackage{float}
\usepackage{enumerate}
\usepackage[top=2.5cm, bottom=2.5cm, left=1.25cm, right=1.25cm]{geometry}

\begin{document}
\pagestyle{empty}

\begin{center}
  \begin{tabular}{ccc}
    \begin{tabular}{c}
      \includegraphics[scale=0.25]{../../biblioteca/imagem/brasao-de-armas-brasil} \\
    \end{tabular} & 
    \begin{tabular}{c}
      Ministério da Educação \\
      Universidade Federal dos Vales do Jequitinhonha e Mucuri \\
      Faculdade de Ciências Sociais, Aplicadas e Exatas - FACSAE \\
      Departamento de Ciências Exatas - DCEX \\
      Disciplina: Introdução à Ciência da Computação \quad Semestre: 2024/1\\
      Prof. Dr. Luiz C. M. de Aquino\\
    \end{tabular} &
    \begin{tabular}{c}
      \includegraphics[scale=0.25]{../../biblioteca/imagem/logo-ufvjm} \\
    \end{tabular}
  \end{tabular}
\end{center}

\begin{center}
  \textbf{Lista VI}
\end{center}

\begin{enumerate}
  \item Faça um programa que permita o usuário digitar 
  os elementos de uma matriz $3\times 3$. Em seguida, o programa
  deve imprimir o resultado da soma dos elementos de cada coluna
  dessa matriz.

  \item Faça um programa que determine se uma matriz $4\times 4$
  é antissimétrica.
  
  \item Faça um programa que calcule o determinante de uma matriz
  $3\times 3$.  
 
  \item Faça um programa que determine o maior elemento da 
  coluna que contém o menor elemento de uma matriz $4\times 4$.

\end{enumerate}

\end{document}