\documentclass[12pt,a4paper]{article}
\usepackage[utf8]{inputenc}
\usepackage[brazil]{babel}
\usepackage{graphicx}
\usepackage{amssymb, amsfonts, amsmath}
\usepackage{float}
\usepackage{enumerate}
\usepackage[top=2.5cm, bottom=2.5cm, left=1.25cm, right=1.25cm]{geometry}

\begin{document}
\pagestyle{empty}

\begin{center}
  \begin{tabular}{ccc}
    \begin{tabular}{c}
      \includegraphics[scale=0.25]{../../biblioteca/imagem/brasao-de-armas-brasil} \\
    \end{tabular} & 
    \begin{tabular}{c}
      Ministério da Educação \\
      Universidade Federal dos Vales do Jequitinhonha e Mucuri \\
      Faculdade de Ciências Sociais, Aplicadas e Exatas - FACSAE \\
      Departamento de Ciências Exatas - DCEX \\
      Disciplina: Introdução à Ciência da Computação \quad Semestre: 2024/1\\
      Prof. Dr. Luiz C. M. de Aquino\\
    \end{tabular} &
    \begin{tabular}{c}
      \includegraphics[scale=0.25]{../../biblioteca/imagem/logo-ufvjm} \\
    \end{tabular}
  \end{tabular}
\end{center}

\begin{center}
  \textbf{Lista III}
\end{center}

\begin{enumerate}
%  \item Faça um programa que fará, sequencialmente, cinco perguntas sobre 
%  um crime:
%
%  -- ``Telefonou para a vítima?''
%  
%  -- ``Esteve no local do crime?''
%  
%  -- ``Mora perto da vítima?''
%
%  -- ``Devia para a vítima?''
%  
%  --  ``Já trabalhou com a vítima?''
%  
%  A resposta de cada pergunta deve ser armazenada em um vetor com 5 posições.
%  Considere como resposta 1 para ``Sim'' e 0 para ``Não''. Ao final de todas
%  as perguntas, o programa deve emitir uma classificação sobre a participação
%  da pessoa no crime. Se a pessoa responder ``Sim'' a 2 perguntas ela deve
%  ser classificada como ``Suspeita''; se responder entre 3 e 4 ``Sim'' deve ser classificada
%  como ``Cúmplice''; se responder ``Sim'' a 5 deve ser classificada como ``Assassino''.
%  Caso contrário, ele será classificado como "Inocente".
  \item Faça um programa que leia um vetor com 10 números inteiros.
  Em seguida, determine a quantidade de números pares e ímpares
  nesse vetor.

  \item Faça um programa que leia um vetor com 10 números inteiros.
  Em seguida, identifique se existem valores repetidos nesse vetor
  e os escreva na tela.
  
  \item Suponha que foi dado o vetor $\texttt{v} = [a_0,\, a_1,\, a_2, \ldots,\, a_{9}]$.
  Faça um programa que calcule o resultado do somatório:
  $$\sum_{i = 0}^9 a_i^2$$
  
  \item Faça um programa que leia 10 números reais e armazene em um vetor.
  Em seguida, calcule o desvio padrão nesse conjunto de números. Observação:
  considerando que o vetor seja \texttt{v}, o desvio padrão $\sigma$ é 
  calculado pela expressão:
  $$\sigma = \sqrt{\frac{1}{10} \sum_{i = 0}^9 (\texttt{v}[i] - m)^2}\textrm{,}$$
  onde $m$ é a média dos valores do vetor.
\end{enumerate}

\end{document}