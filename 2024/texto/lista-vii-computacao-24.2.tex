\documentclass[12pt,a4paper]{article}
\usepackage[utf8]{inputenc}
\usepackage[brazil]{babel}
\usepackage{graphicx}
\usepackage{amssymb, amsfonts, amsmath}
\usepackage{float}
\usepackage{enumerate}
\usepackage[top=2.5cm, bottom=2.5cm, left=1.25cm, right=1.25cm]{geometry}

\begin{document}
\pagestyle{empty}

\begin{center}
  \begin{tabular}{ccc}
    \begin{tabular}{c}
      \includegraphics[scale=0.25]{../../biblioteca/imagem/brasao-de-armas-brasil} \\
    \end{tabular} & 
    \begin{tabular}{c}
      Ministério da Educação \\
      Universidade Federal dos Vales do Jequitinhonha e Mucuri \\
      Faculdade de Ciências Sociais, Aplicadas e Exatas - FACSAE \\
      Departamento de Ciências Exatas - DCEX \\
      Disciplina: Introdução à Ciência da Computação \quad Semestre: 2024/2\\
      Prof. Dr. Luiz C. M. de Aquino\\
    \end{tabular} &
    \begin{tabular}{c}
      \includegraphics[scale=0.25]{../../biblioteca/imagem/logo-ufvjm} \\
    \end{tabular}
  \end{tabular}
\end{center}

\begin{center}
  \textbf{Lista VII}
\end{center}

\begin{enumerate}
   
  \item Faça um programa que permita o usuário digitar 
  os elementos de uma matriz $3\times 3$. Em seguida, o programa
  deve calcular o traço dessa matriz. (Obs.: o traço de uma matriz 
  quadrada é a soma de todos os elementos na sua diagonal principal.)

  \item Faça um programa que determine se uma matriz $4\times 4$
  é antissimétrica.
  
%  \item Faça um programa que calcule o determinante de uma matriz
%  $3\times 3$.
 
  \item Faça um programa que determine o maior elemento da 
  coluna que contém o menor elemento de uma matriz $4\times 4$.

%  \item Faça um programa que recebe uma matriz $4\times 4$ e um número
%  $n$. O programa deve imprimir na tela a posição (linha e coluna) que o
%  número $n$ ocupa na matriz. Caso $n$ não esteja na matriz, o programa deve
%  imprimir ``número não encontrado''.
  
  \item Faça um programa que recebe duas matrizes $4\times 4$ e crie uma
  terceira matriz com o maior elemento de cada posição das matrizes dadas.
  
  \item Suponha que 5 alunos fizeram uma prova de múltipla escolha 
  com 10 questões. Faça um programa que recebe uma matriz $5\times 10$
  representando as respostas desses alunos para a prova. Considere que 
  a resposta de cada questão seja representada pelas opções 1, 2, 3, 4
  ou 5. O seu programa deve receber também um vetor com 10 posições
  contendo o gabarito da prova. Desse modo, o programa deve comparar as
  respostas de cada aluno com o gabarito e calcular sua respectiva nota.
  Considere que cada questão vale 10,0 pontos.

\end{enumerate}

\end{document}