\documentclass[12pt,a4paper]{article}
\usepackage[utf8]{inputenc}
\usepackage[brazil]{babel}
\usepackage{graphicx}
\usepackage{amssymb, amsfonts, amsmath}
\usepackage{float}
\usepackage{enumerate}
\usepackage[top=2.5cm, bottom=2.5cm, left=1.25cm, right=1.25cm]{geometry}

\begin{document}
\pagestyle{empty}

\begin{center}
  \begin{tabular}{ccc}
    \begin{tabular}{c}
      \includegraphics[scale=0.25]{../../biblioteca/imagem/brasao-de-armas-brasil} \\
    \end{tabular} & 
    \begin{tabular}{c}
      Ministério da Educação \\
      Universidade Federal dos Vales do Jequitinhonha e Mucuri \\
      Faculdade de Ciências Sociais, Aplicadas e Exatas - FACSAE \\
      Departamento de Ciências Exatas - DCEX \\
      Disciplina: Introdução à Ciência da Computação \quad Semestre: 2024/2\\
      Prof. Dr. Luiz C. M. de Aquino\\
    \end{tabular} &
    \begin{tabular}{c}
      \includegraphics[scale=0.25]{../../biblioteca/imagem/logo-ufvjm} \\
    \end{tabular}
  \end{tabular}
\end{center}

\begin{center}
  \textbf{Lista IV}
\end{center}

\begin{enumerate}
  \item Desenvolva um gerador de tabuada, capaz de gerar a tabuada de qualquer número inteiro
  entre 1 a 10. O usuário deve informar de qual número ele deseja ver a tabuada. A saída deve
  ser conforme o exemplo abaixo.

    \vspace{0.5cm}
    Tabuada de 5:
    
    5 x 1 = 5
    
    5 x 2 = 10
    
    5 x 3 = 15
    
    5 x 4 = 20
    
    5 x 5 = 25

    5 x 6 = 30
    
    5 x 7 = 35
    
    5 x 8 = 40
    
    5 x 9 = 45
        
    5 x 10 = 50
    
    \vspace{0.5cm}


  \item Faça um programa que leia um número natural $n \neq 0$ e imprima na tela o valor da soma:
  $$S = \frac{1}{2} + \frac{2}{3} + \frac{3}{4} + \ldots + \frac{n - 1}{n}$$
    
  \item Faça um programa para determinar o número de dígitos de um número
   inteiro positivo informado.
  
  \item Numa eleição existem três candidatos. Faça um programa que peça o número total de eleitores
  e em seguida peça para cada um deles o voto. No final, o programa deve mostrar o número de votos
  de cada candidato e qual foi o candidato eleito.
  
  \item Faça um programa que leia uma quantidade indeterminada de números positivos e conte
  quantos deles estão nos seguintes intervalos: [0-25], [26-50], [51-75] e [76-100].
  A entrada de dados deverá terminar quando for lido um número negativo.   

\end{enumerate}

\end{document}