\documentclass[12pt,a4paper]{article}
\usepackage[utf8]{inputenc}
\usepackage[brazil]{babel}
\usepackage{graphicx}
\usepackage{amssymb, amsfonts, amsmath}
\usepackage{float}
\usepackage{enumerate}
\usepackage[top=2.5cm, bottom=2.5cm, left=1.25cm, right=1.25cm]{geometry}

\begin{document}
\pagestyle{empty}

\begin{center}
  \begin{tabular}{ccc}
    \begin{tabular}{c}
      \includegraphics[scale=0.25]{../../biblioteca/imagem/brasao-de-armas-brasil} \\
    \end{tabular} & 
    \begin{tabular}{c}
      Ministério da Educação \\
      Universidade Federal dos Vales do Jequitinhonha e Mucuri \\
      Faculdade de Ciências Sociais, Aplicadas e Exatas - FACSAE \\
      Departamento de Ciências Exatas - DCEX \\
      Disciplina: Introdução à Ciência da Computação \quad Semestre: 2024/2\\
      Prof. Dr. Luiz C. M. de Aquino\\
    \end{tabular} &
    \begin{tabular}{c}
      \includegraphics[scale=0.25]{../../biblioteca/imagem/logo-ufvjm} \\
    \end{tabular}
  \end{tabular}
\end{center}

\begin{center}
  \textbf{Lista V}
\end{center}

\begin{enumerate}
  \item Faça um programa que leia um número natural $n$ e crie um vetor $\texttt{v}$
  com os algarismos desse número. Por exemplo, se $n = 1984$, então
  $\texttt{v} = [1,\,9,\,8,\,4]$.
  
  \item Suponha que \texttt{u} e \texttt{v} são os vetores formados com os algarismos de
  dois números naturais, respectivamente, $a$ e $b$. Faça um programa que calcule a soma
  $a + b$ e armazene o resultado em um vetor \texttt{s}. Por exemplo, se $a = 1245$ e $b = 382$,
  então $\texttt{u} = [1,\,2,\,4,\,5]$, $\texttt{v} = [3,\,8,\,2]$ e
  $\texttt{s} = [1,\,6,\,2,\,7]$.

  \item Faça um programa que recebe um vetor $\texttt{v}$ com 10 números inteiros e procure nele
  um certo número inteiro $n$ dado. Caso $n$ seja encontrado no vetor, o programa deve imprimir
  a posição que ele está no vetor. Caso contrário, o programa deve imprimir a mensagem que 
  $n$ não foi encontrado.

  \item Faça um programa que leia 10 números reais e armazene em um vetor.
  Em seguida, calcule o desvio padrão nesse conjunto de números. Observação:
  considerando que o vetor seja \texttt{v}, o desvio padrão $\sigma$ é 
  calculado pela expressão:
  $$\sigma = \sqrt{\frac{1}{10} \sum_{i = 0}^9 (\texttt{v}[i] - m)^2}\textrm{,}$$
  onde $m$ é a média dos valores do vetor.
\end{enumerate}

\end{document}