\documentclass[12pt,a4paper]{article}
\usepackage[utf8]{inputenc}
\usepackage[brazil]{babel}
\usepackage{graphicx}
\usepackage{amssymb, amsfonts, amsmath}
\usepackage{float}
\usepackage{enumerate}
\usepackage[top=2.5cm, bottom=2.5cm, left=1.25cm, right=1.25cm]{geometry}

\begin{document}
\pagestyle{empty}

\begin{center}
  \begin{tabular}{ccc}
    \begin{tabular}{c}
      \includegraphics[scale=0.25]{../../biblioteca/imagem/brasao-de-armas-brasil} \\
    \end{tabular} & 
    \begin{tabular}{c}
      Ministério da Educação \\
      Universidade Federal dos Vales do Jequitinhonha e Mucuri \\
      Faculdade de Ciências Sociais, Aplicadas e Exatas - FACSAE \\
      Departamento de Ciências Exatas - DCEX \\
      Disciplina: Introdução à Ciência da Computação \quad Semestre: 2023/1\\
      Prof. Dr. Luiz C. M. de Aquino\\
    \end{tabular} &
    \begin{tabular}{c}
      \includegraphics[scale=0.25]{../../biblioteca/imagem/logo-ufvjm} \\
    \end{tabular}
  \end{tabular}
\end{center}

\begin{center}
  \textbf{Lista I}
\end{center}

\begin{enumerate}
  \item Faça um programa que leia três números e mostre o maior e o menor deles.
  \item Faça um programa que leia três números e mostre-os em ordem decrescente.
  \item Faça um programa para a leitura de duas notas parciais de um aluno. O programa deve 
  calcular a média alcançada pelo aluno e apresentar:
    \begin{itemize}  
      \item a mensagem ``Aprovado'', se a média alcançada for maior ou igual a sete;
      \item a mensagem ``Reprovado'', se a média for menor do que sete;
      \item a mensagem ``Aprovado com Distinção'', se a média for igual a dez.     
    \end{itemize}
  \item Faça um programa que leia um número e exiba o dia correspondente da semana (1- Domingo,
  2- Segunda, etc.). Se digitar um valor acima de 7, então deve aparecer "valor inválido".
  
  \item Escreva um programa que imprima na tela os $n$ primeiros termos da sequência de Fibonacci.
  Por exemplo, os 10 primeiros termos são:
  
  \begin{center}
    1, 1, 2, 3, 5, 8, 13, 21, 34, 55
  \end{center}
  
  \item Faça um programa que peça uma nota, entre zero e dez. Mostre uma mensagem caso o valor seja
  inválido e continue pedindo até que o usuário informe um valor válido.
  
  \item Faça um programa que leia 10 números e informe o maior deles.
  
  \item Faça um programa que leia 10 números e informe o menor deles.

  \item Faça um programa que peça um número inteiro e determine se ele é ou não um número primo.
  
  \item Faça um programa que calcule o número médio de alunos por turma. Para isto,
  peça a quantidade de turmas e a quantidade de alunos para cada turma. As turmas não podem ter
  mais de 40 alunos.
  
\end{enumerate}
\end{document}