\documentclass[12pt,a4paper]{article}
\usepackage[utf8]{inputenc}
\usepackage[brazil]{babel}
\usepackage{graphicx}
\usepackage{amssymb, amsfonts, amsmath}
\usepackage{float}
\usepackage{enumerate}
\usepackage[top=2.5cm, bottom=2.5cm, left=1.25cm, right=1.25cm]{geometry}

\begin{document}
\pagestyle{empty}

\begin{center}
  \begin{tabular}{ccc}
    \begin{tabular}{c}
      \includegraphics[scale=0.25]{../../biblioteca/imagem/brasao-de-armas-brasil} \\
    \end{tabular} & 
    \begin{tabular}{c}
      Ministério da Educação \\
      Universidade Federal dos Vales do Jequitinhonha e Mucuri \\
      Faculdade de Ciências Sociais, Aplicadas e Exatas - FACSAE \\
      Departamento de Ciências Exatas - DCEX \\
      Disciplina: Introdução à Ciência da Computação \quad Semestre: 2022/2\\
      Prof. Dr. Luiz C. M. de Aquino\\
      Aluno(a):\rule{6cm}{0.1mm} \quad Data: \rule{0.5cm}{0.1mm}/\rule{0.5cm}{0.1mm}/\rule{1cm}{0.1mm}\\
    \end{tabular} &
    \begin{tabular}{c}
      \includegraphics[scale=0.25]{../../biblioteca/imagem/logo-ufvjm} \\
    \end{tabular}
  \end{tabular}
\end{center}

\begin{center}
 \textbf{Avaliação II}
\end{center}

\textbf{Instruções}
\begin{itemize}
 \item Todas as justificativas necessárias na solução de cada questão devem 
 estar presentes nesta avaliação;
 \item As respostas finais de cada questão devem estar escritas de caneta;
 \item Esta avaliação tem um total de 35,0 pontos.
\end{itemize}

\begin{enumerate}
  \item \textbf{[7,0 pontos]} Faça um programa que leia um conjunto de 15 números
  reais, armazenando-os em um vetor. Em seguida, calcule a soma e o produto desses
  valores.
  
  \item \textbf{[7,0 pontos]} Faça um programa que recebe um vetor com 15 números
  inteiros. Verifique se existem valores iguais nesse vetor e os escreva na tela.
  
  \item \textbf{[7,0 pontos]} Faça um programa que recebe 15 números reais em 
  um vetor e calcule o desvio padrão nesse conjunto
  de números. Observação:
  considerando que o vetor seja \texttt{v}, o desvio padrão $\sigma$ é 
  calculado pela expressão:
  $$\sigma = \sqrt{\frac{1}{15} \sum_{i = 0}^{14} (\texttt{v}[i] - m)^2}\textrm{,}$$
  onde $m$ é a média dos valores do vetor.
  
  \item \textbf{[7,0 pontos]} Faça um programa que determine a interseção entre
  dois conjuntos com 15 elementos cada. Cada conjunto deve ser armazenado como
  um vetor.

  \item \textbf{[7,0 pontos]} Faça um programa que leia um número natural $n$ 
  e crie um vetor $\texttt{v}$ com os algarismos desse número. Por exemplo,
  se $n = 1984$, então $\texttt{v} = [1,\,9,\,8,\,4]$.

\end{enumerate}

\end{document}