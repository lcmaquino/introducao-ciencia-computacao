\documentclass[12pt,a4paper]{article}
\usepackage[utf8]{inputenc}
\usepackage[brazil]{babel}
\usepackage{graphicx}
\usepackage{amssymb, amsfonts, amsmath}
\usepackage{float}
\usepackage{enumerate}
\usepackage[top=2.5cm, bottom=2.5cm, left=1.25cm, right=1.25cm]{geometry}

\begin{document}
\pagestyle{empty}

\begin{center}
  \begin{tabular}{ccc}
    \begin{tabular}{c}
      \includegraphics[scale=0.25]{../../biblioteca/imagem/brasao-de-armas-brasil} \\
    \end{tabular} & 
    \begin{tabular}{c}
      Ministério da Educação \\
      Universidade Federal dos Vales do Jequitinhonha e Mucuri \\
      Faculdade de Ciências Sociais, Aplicadas e Exatas - FACSAE \\
      Departamento de Ciências Exatas - DCEX \\
      Disciplina: Introdução à Ciência da Computação \quad Semestre: 2022/2\\
      Prof. Dr. Luiz C. M. de Aquino\\
    \end{tabular} &
    \begin{tabular}{c}
      \includegraphics[scale=0.25]{../../biblioteca/imagem/logo-ufvjm} \\
    \end{tabular}
  \end{tabular}
\end{center}

\begin{center}
  \textbf{Lista IV}
\end{center}

\begin{enumerate}
  \item  Faça um programa que determine a interseção entre dois conjuntos com 
  10 elementos cada. Cada conjunto deve ser armazenado como um vetor.

  \item  Faça um programa que determine a subtração entre dois conjuntos com 
  10 elementos cada. Cada conjunto deve ser armazenado como um vetor. (Observação:
  dados os conjuntos $A$ e $B$, a subtração $A - B$ é definida como sendo o conjunto
  $A - B = \{x\,|\,x\in A \textrm{ e } x\not\in B\}$.)

  \item Faça um programa que leia um número natural $n$ e crie um vetor $\texttt{v}$
  com os algarismos desse número. Por exemplo, se $n = 1984$, então
  $\texttt{v} = [1,\,9,\,8,\,4]$.
    
  \item Suponha que \texttt{u} e \texttt{v} são os vetores formados com os algarismos de
  dois números naturais, respectivamente, $a$ e $b$. Faça um programa que calcule a soma
  $a + b$ e armazene o resultado em um vetor \texttt{s}. Por exemplo, se $a = 1245$ e $b = 382$,
  então $\texttt{u} = [1,\,2,\,4,\,5]$, $\texttt{v} = [3,\,8,\,2]$ e
  $\texttt{s} = [1,\,6,\,2,\,7]$.

  \item Faça um programa que recebe um vetor $\texttt{v}$ com 10 números inteiros e coloque
  em ordem crescente.

\end{enumerate}

\end{document}