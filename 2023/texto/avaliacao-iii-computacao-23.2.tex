\documentclass[12pt,a4paper]{article}
\usepackage[utf8]{inputenc}
\usepackage[brazil]{babel}
\usepackage{graphicx}
\usepackage{amssymb, amsfonts, amsmath}
\usepackage{float}
\usepackage{enumerate}
\usepackage[top=2.5cm, bottom=2.5cm, left=1.25cm, right=1.25cm]{geometry}

\begin{document}
\pagestyle{empty}

\begin{center}
  \begin{tabular}{ccc}
    \begin{tabular}{c}
      \includegraphics[scale=0.25]{../../biblioteca/imagem/brasao-de-armas-brasil} \\
    \end{tabular} & 
    \begin{tabular}{c}
      Ministério da Educação \\
      Universidade Federal dos Vales do Jequitinhonha e Mucuri \\
      Faculdade de Ciências Sociais, Aplicadas e Exatas - FACSAE \\
      Departamento de Ciências Exatas - DCEX \\
      Disciplina: Introdução à Ciência da Computação \quad Semestre: 2023/2\\
      Prof. Dr. Luiz C. M. de Aquino\\
    \end{tabular} &
    \begin{tabular}{c}
      \includegraphics[scale=0.25]{../../biblioteca/imagem/logo-ufvjm} \\
    \end{tabular}
  \end{tabular}
\end{center}

\begin{center}
 \textbf{Avaliação III}
\end{center}

\textbf{Instruções}
\begin{itemize}
 \item Todas as justificativas necessárias na solução de cada questão devem 
 estar presentes nesta avaliação;
 \item As respostas finais de cada questão devem estar escritas de caneta;
 \item Esta avaliação tem um total de 35,0 pontos.
\end{itemize}

\begin{enumerate}
  \item  \textbf{[7,0 pontos]} Faça uma função que recebe como argumento um número
  natural $n$ e retorna a soma de todos os números naturais até $n$.

  \item  \textbf{[7,0 pontos]} Faça uma função que recebe como argumento um número
  natural $n$ e retorna um vetor com todos os divisores de $n$.
  
  \item \textbf{[7,0 pontos]} Faça uma função que recebe como argumentos dois números
  naturais $m$ e $n$ e retorna o resultado da divisão inteira de $m$ por $n$.
  Na sua função use apenas as operações aritméticas de soma ou subtração.

  \item \textbf{[7,0 pontos]} Faça uma função que recebe como argumento um vetor
  com números reais e retorna a média entre esses números.
 
  \item \textbf{[7,0 pontos]} Considere a sequência $\{a_1,\,a_2,\,a_3,\,\cdots\}$ cujo $n$-ésimo
  termo é definido por:
  $$a_n = 
      \begin{cases}
      2,\,\textrm{ se } n < 3 \\
      3a_{n-1} + 2a_{n-2},\,\textrm{ se } n \geq 3
      \end{cases}
  $$
  
  Faça uma função que recebe como argumento um número natural $k \geq 1$ e retorna o valor do termo $a_k$.

\end{enumerate}

\end{document}