\documentclass[12pt,a4paper]{article}
\usepackage[utf8]{inputenc}
\usepackage[brazil]{babel}
\usepackage{graphicx}
\usepackage{amssymb, amsfonts, amsmath}
\usepackage{float}
\usepackage{enumerate}
\usepackage[top=2.5cm, bottom=2.5cm, left=1.25cm, right=1.25cm]{geometry}

\begin{document}
\pagestyle{empty}

\begin{center}
  \begin{tabular}{ccc}
    \begin{tabular}{c}
      \includegraphics[scale=0.25]{../../biblioteca/imagem/brasao-de-armas-brasil} \\
    \end{tabular} & 
    \begin{tabular}{c}
      Ministério da Educação \\
      Universidade Federal dos Vales do Jequitinhonha e Mucuri \\
      Faculdade de Ciências Sociais, Aplicadas e Exatas - FACSAE \\
      Departamento de Ciências Exatas - DCEX \\
      Disciplina: Introdução à Ciência da Computação \quad Semestre: 2023/2\\
      Prof. Dr. Luiz C. M. de Aquino\\
      Aluno(a):\rule{6cm}{0.1mm} \quad Data: \rule{0.5cm}{0.1mm}/\rule{0.5cm}{0.1mm}/\rule{1cm}{0.1mm}\\
    \end{tabular} &
    \begin{tabular}{c}
      \includegraphics[scale=0.25]{../../biblioteca/imagem/logo-ufvjm} \\
    \end{tabular}
  \end{tabular}
\end{center}

\begin{center}
 \textbf{Avaliação I}
\end{center}

\textbf{Instruções}
\begin{itemize}
 \item Todas as justificativas necessárias na solução de cada questão devem 
 estar presentes nesta avaliação;
 \item As respostas finais de cada questão devem estar escritas de caneta;
 \item Esta avaliação tem um total de 30,0 pontos.
\end{itemize}

\begin{enumerate}
  \item \textbf{[6,0 pontos]} Faça um programa para a leitura de quatro notas parciais de um aluno.
  O programa deve calcular a média alcançada pelo aluno e apresentar a seguinte mensagem:
    \begin{itemize}
      \item ``Aprovado com Distinção'', se a média for igual a 100.
      \item ``Aprovado'', se a média for maior ou igual a 70 e menor do que 100;
      \item ``Fazer Exame Final'', se a média for maior ou igual a 40 e menor do que 70;
      \item ``Reprovado'', se a média for menor do que 40;
    \end{itemize}
  
  \item \textbf{[6,0 pontos]} Escreva um programa que imprima na tela os $n$ primeiros termos
  da sequência de Fibonacci. Por exemplo, os 10 primeiros termos são:
  
  \begin{center}
    1, 1, 2, 3, 5, 8, 13, 21, 34, 55
  \end{center}
  
  \item \textbf{[6,0 pontos]} Faça um programa que leia dois números naturais $n$ e $m$.
  Em seguida, o programa deve imprimir a soma entre todos os números ímpares no intervalo
  $[n,\, m]$.

  \item \textbf{[6,0 pontos]} Faça um programa que peça um número inteiro e determine se
  ele é ou não um número primo.

  \item \textbf{[6,0 pontos]} Faça um programa para determinar o número de dígitos
  de um número inteiro positivo informado.

\end{enumerate}

\end{document}