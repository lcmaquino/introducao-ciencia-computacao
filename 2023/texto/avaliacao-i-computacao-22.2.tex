\documentclass[12pt,a4paper]{article}
\usepackage[utf8]{inputenc}
\usepackage[brazil]{babel}
\usepackage{graphicx}
\usepackage{amssymb, amsfonts, amsmath}
\usepackage{float}
\usepackage{enumerate}
\usepackage[top=2.5cm, bottom=2.5cm, left=1.25cm, right=1.25cm]{geometry}

\begin{document}
\pagestyle{empty}

\begin{center}
  \begin{tabular}{ccc}
    \begin{tabular}{c}
      \includegraphics[scale=0.25]{../../biblioteca/imagem/brasao-de-armas-brasil} \\
    \end{tabular} & 
    \begin{tabular}{c}
      Ministério da Educação \\
      Universidade Federal dos Vales do Jequitinhonha e Mucuri \\
      Faculdade de Ciências Sociais, Aplicadas e Exatas - FACSAE \\
      Departamento de Ciências Exatas - DCEX \\
      Disciplina: Introdução à Ciência da Computação \quad Semestre: 2022/2\\
      Prof. Dr. Luiz C. M. de Aquino\\
      Aluno(a):\rule{6cm}{0.1mm} \quad Data: \rule{0.5cm}{0.1mm}/\rule{0.5cm}{0.1mm}/\rule{1cm}{0.1mm}\\
    \end{tabular} &
    \begin{tabular}{c}
      \includegraphics[scale=0.25]{../../biblioteca/imagem/logo-ufvjm} \\
    \end{tabular}
  \end{tabular}
\end{center}

\begin{center}
 \textbf{Avaliação I}
\end{center}

\textbf{Instruções}
\begin{itemize}
 \item Todas as justificativas necessárias na solução de cada questão devem 
 estar presentes nesta avaliação;
 \item As respostas finais de cada questão devem estar escritas de caneta;
 \item Esta avaliação tem um total de 30,0 pontos.
\end{itemize}

\begin{enumerate}
  \item \textbf{[6,0 pontos]} Faça um programa para a leitura de quatro notas parciais de um aluno.
  O programa deve calcular a média alcançada pelo aluno e apresentar a mensagem:
    \begin{itemize}
      \item ``Aprovado com Distinção'', se a média for igual a dez.
      \item ``Aprovado'', se a média for maior ou igual a sete e menor do que 10;
      \item ``Recuperação'', se a média for maior ou igual a quatro e menor do que sete;
      \item ``Reprovado'', se a média for menor do que quatro;
    \end{itemize}
  
  \item \textbf{[6,0 pontos]} Faça um programa que calcule o número médio de alunos por turma.
  Para isto, peça a quantidade de turmas e a quantidade de alunos para cada turma. 
  As turmas não podem ter mais de 30 alunos.
  
  \item \textbf{[6,0 pontos]} Faça um programa que leia dois números naturais $n$ e $m$.
  Em seguida, o programa deve imprimir a soma entre todos os números ímpares no intervalo
  $[n,\, m]$.

  \item \textbf{[6,0 pontos]} Numa eleição existem cinco candidatos. Faça um 
  programa que peça o número total de eleitores e em seguida peça para cada um deles
  o voto. No final, o programa deve mostrar o número de votos de cada candidato e
  qual foi o candidato eleito.

  \item \textbf{[6,0 pontos]} Faça um programa que leia uma quantidade indeterminada 
  de números inteiros positivos e conte quantos deles estão nos seguintes intervalos:
  [0-20], [21-40], [41-60] e [61-100]. A entrada de dados deverá terminar 
  quando for lido um número negativo.

\end{enumerate}

\end{document}